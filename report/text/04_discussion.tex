\clearpage
\section{Discussão Crítica}

A análise dos resultados revela um comportamento dual do modelo LSTM, que varia drasticamente com a natureza
do \emph{dataset} utilizado.

No \emph{dataset} de 15 minutos, que exibe um tráfego relativamente estável, o modelo foi capaz de aprender a
tendência geral e a sazonalidade de 60 segundos identificada na EDA. A previsão, embora mais suave que os
dados reais, seguiu a trajetória da série. Isso demonstra a capacidade da arquitetura LSTM de capturar
padrões de curto e médio prazo quando o comportamento da rede não apresenta variações extremas.

Em contraste, no \emph{dataset} de 9 horas, o desempenho do modelo foi paradoxal. Por um lado, ele se tornou
um preditor extremamente preciso da linha de base do tráfego. Os valores de MSE foram ordens de magnitude
menores, pois o modelo aprendeu a prever com exatidão os longos períodos de baixa atividade. Por outro lado,
ele falhou completamente em antecipar os \emph{bursts} de tráfego súbitos e de alta magnitude. O modelo
tratou esses picos como ruído imprevisível, mantendo sua previsão próxima da média histórica e não
conseguindo se adaptar a esses eventos de "cisne negro".

Essa limitação é uma consequência direta da simplicidade da arquitetura do modelo. Uma única camada LSTM,
embora eficaz para padrões regulares, não possui a capacidade de modelar a natureza inerentemente "ruidosa" e
multifacetada do tráfego de internet, que é influenciado por uma infinidade de fatores, como o controle de
congestionamento do TCP, a agregação de milhares de fluxos independentes e eventos externos imprevisíveis.

\subsection{Conclusão}

Em síntese, o projeto demonstrou com sucesso a viabilidade de utilizar uma arquitetura LSTM simples para
modelar e prever o volume de tráfego de rede. A análise exploratória foi fundamental para guiar a seleção de
hiperparâmetros, como a janela de \emph{look-back}, e para interpretar o comportamento do modelo.

A principal vantagem da abordagem foi a capacidade do modelo de aprender tendências e padrões sazonais de
longo prazo, resultando em uma previsão muito precisa da linha de base do tráfego. No entanto, sua principal
limitação foi a incapacidade de prever os \emph{bursts} de curta duração, que são de grande importância para
aplicações como detecção de anomalias ou planejamento de capacidade em tempo real.

Para trabalhos futuros, a performance poderia ser aprimorada com a exploração de arquiteturas mais complexas.
Modelos com múltiplas camadas LSTM, LSTMs bidirecionais, ou a incorporação de mecanismos de atenção poderiam
capacitar o modelo a aprender relações temporais mais sofisticadas. Além disso, a engenharia de
\emph{features}, incluindo variáveis como a distribuição de protocolos por segundo ou a hora do dia como uma
entrada explícita, poderia fornecer ao modelo um contexto mais rico, melhorando sua capacidade de prever os
picos de tráfego e tornando-o uma ferramenta mais robusta e prática para o gerenciamento de redes.
